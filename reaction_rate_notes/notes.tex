\documentclass{article}
\usepackage{amsmath}
\usepackage{amsfonts}
\usepackage{amssymb}
\usepackage[margin=1in]{geometry}

\begin{document}
	\title{Notes on Reaction Rates}
	\author{Kevin Ingles}
	\date{}
	\maketitle
	
	A system of reaction rate equations can compactly be written as
	\begin{equation}
		\label{eq:general-rr-1}
		\frac{d \mathfrak n_i}{dt}
		=
		\sum_J \left\lbrack 
			\mathcal G_{J\to i} \prod_{j\in J} \mathfrak n_j
			-
			\mathcal L_{i\to J} \mathfrak n_i
		\right\rbrack.
	\end{equation}
	where $N_i$ is the number for particle species $i$, $\mathcal G_{J\to i}$ and $\mathcal J_{i\to J}$ represent the reaction rates for the \emph{gain} and \emph{loss} terms, respectively, and $J$ is an index set containing the indices for the particle species participating in the reaction.
	
	The density, $\mathfrak n$, can be expressed as a ration of the number of particles $N$ in a system of volume $V$; both $N$ and $V$ can be time dependent.
	Applying the chain rule gives
	\begin{equation}
		\frac{dN}{dt}
		=
		V(t) \frac{d\mathfrak n}{dt} + \mathfrak n(t) \frac{d V}{dt},
		\qquad
		\frac{d \mathfrak n_i}{dt}
		=
		\sum_J \left\lbrack 
		\mathcal G_{J\to i} \prod_{j\in J} \mathfrak n_j
		-
		\mathcal L_{i\to J} \mathfrak n_i
		\right\rbrack.
	\end{equation}
	The general reaction rate, Eq.~\eqref{eq:general-rr-1}, then needs to be modified to take into account the changing volume.
	A convenient trick is to choose a special spices $i^\ast$ and evolve the rations $x_j = \mathfrak n_j / \mathfrak n_{i^\ast}$.
	For all 
	% \begin{equation}
\end{document}